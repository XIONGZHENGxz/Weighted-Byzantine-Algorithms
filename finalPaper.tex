\documentclass[conference]{IEEEtran}
\IEEEoverridecommandlockouts
% The preceding line is only needed to identify funding in the first footnote. If that is unneeded, please comment it out.
\usepackage{cite}
\usepackage{amsmath,amssymb,amsfonts}
\usepackage{graphicx}
\usepackage{textcomp}
\usepackage{algorithm}
\usepackage[]{algorithmic}
\usepackage{amsthm}
\newtheorem{t1}{Theorem}
\newtheorem{l1}{Lemma}
\newtheorem{claim*}{Claim}

%\usepackage{algpseudocode}
\def\BibTeX{{\rm B\kern-.05em{\sc i\kern-.025em b}\kern-.08em
    T\kern-.1667em\lower.7ex\hbox{E}\kern-.125emX}}
\begin{document}

\title{Gradecast for Weighted Byzantine Agreement\\
}

\author{\IEEEauthorblockN{Xiong Zheng}
\IEEEauthorblockA{\textit{Department of Electrical and Computer Engineering} \\
\textit{The University of Texas at Austin}\\
Austin, USA \\
zhengxiongtym@utexas.edu}
\and
\IEEEauthorblockN{Haowei Sun}
\IEEEauthorblockA{\textit{Department of Electrical and Computer Engineering} \\
\textit{The University of Texas at Austin}\\
Austin, USA \\
michelle.shw@utexas.edu}
}

\maketitle

\begin{abstract}
This document is a model and instructions for \LaTeX.
This and the IEEEtran.cls file define the components of your paper [title, text, heads, etc.]. *CRITICAL: Do Not Use Symbols, Special Characters, Footnotes, 
or Math in Paper Title or Abstract.
\end{abstract}

\begin{IEEEkeywords}
component, formatting, style, styling, insert
\end{IEEEkeywords}

\section{Introduction}
The Byzantine Agreement (BA) [1], [2] is a fundamental
problem in distributed computing with extensive literature [3]–
[9]. In the usual set-up, there are N processes required to agree
on a common value, given that at most f of them may show
arbitrary or Byzantine behavior. In real-life applications, there
may be multiple classes of processes. For example, in a mobile
computing scenario, mobile hosts may be more likely to fail
compared to mobile stations. In another example, processes in
the same data center may be more likely to fail together. This
paper defines a weighted version of the Byzantine Agreement
Problem (WBA) and provides lower bounds and algorithms
for it. In WBA, each process Pi is assigned a weight w[i],
such that 0 ≤ w[i] ≤ 1 and the sum of all weights is 1.
The WBA problem requires a protocol to reach consensus
when the total weight of the failed processes is at most ρ.
The weighted version gives some surprising results for the
BA problem. First, even if greater than N=3 processes are
Byzantine, the system can still reach consensus so long as ρ
is less than 1=3. This result is quite useful in the system with
a small set of trusted processes and a large set of less trusted
processes.
Secondly, the message complexity and the number of rounds
required to achieve consensus for the weighted version is
shown to always be less than or equal to those for the unweighted version. Suppose the system must tolerate ρ = f=N
for any integer f such that 0 ≤ f < N=3. It is known 
\section{System Model}
The system model used in this paper is a distributed system
with $N$ processes, $P1...PN$, with a completely connected topology. The underlying system is assumed to be synchronous; i.e.,
there is an upper bound on the message delay and on the duration of actions performed by processes. The model assumes
that processes may fail; but, the underlying communication
system is reliable and satisfies first-in first-out (FIFO) message
ordering. The processes may fail in an arbitrary fashion; in
particular, they may lie and collude with other failed processes
to foil any protocol. The processes that do not fail in any
computation are called correct processes. Assume that there
is a non-negative weight $w[i]$ associated with each process
Pi. All processes in the system have complete knowledge
of weights of all the processes. For simplicity, assume that
weights are normalized; i.e., the sum of all weights is one.
Let $ρ$ be the sum of weights of all failed processes. This paper
assumes that $ρ$ is strictly less than one.The Weighted Byzantine Agreement (WBA) problem can be specified as follows. All processes propose a binary value with the goal of deciding on one common value. Given a
weight assignment to all processes, and the assumption that
the weight of processes that fail during the execution is at most
$ρ$, the WBA problem is to design a protocol that satisfies the
following conditions:
• Agreement: Two correct processes cannot decide on
different values.
• Validity: The value decided must be proposed by some
correct process.
• Termination: All correct processes decide in finite number of steps.
The following lower bounds follow easily from the standard
BA lower bound arguments.
Lemma 1: There is no protocol to solve the WBA problem
for all values of $w$ when ρ $\leq$ 1/3.


\section{Weighted Gradecast Algorithm}
Gradecast[7] is a distributed algorithm that ensures some properties that are similar to those of broadcast. Specifically, in Gradecast there is a sender node $p$ that sends a value $v$ to all other. Each node $q’$s output is a pair hvq, cqi where vq is the value $q$ thinks $p$ has sent and $c_q$ is $q'$s confidence in this value. The Gradecast properties ensure that:

1. If $p$ is non-faulty then $v_q$ = $v$ and $c_q = 2$, for every non-faulty $q$;

2. For every non-faulty nodes $q$, $q'$: if $c_q > 0$ and $c_q' > 0$ then $v_q = v_q'$;

3. $|c_q - c_q'| \leq 1$ for every non-faulty nodes $q$, $q'$.


For the WBA problem,  in order to apply gradecast, we simply need to do some minor modification of the original gradecast algorithm proposed by Michael et. all in [8]. The algorithm uses constant size messages and requires that $\rho < 1/3$. Each process has a preferred value for each round. The modified algorithm is presented in Fig. 1. To show the WEIGHTED-GRADECAST algorithm can solve the WBA problem, we need to show that the above three Gradecast properties hold and the proof for Byzantine consensus in [8] still holds. 
 
\begin{claim*}
if non-faulty node $p$ has $c_q = 2$, then any other non-faulty node $q$ has $c_q \geq 1$ and $v_q = v_p$
\begin{proof}
Since $c_p = 2$, from the grading part of \textbf{Algorithm 1} we know that the total weight $w'$ (for $v_p) \leq 1 - \rho$. There are at most $\rho$ faulty nodes which implies that the weights from non-faulty nodes for value $v_p$ in round 3 is at least $1 - 2*\rho$. Similarly,  the weights from non-faulty Since $\rho < 1/3$, we have $1 - 2*\rho > \rho$. So for any other non-fault node $q$, the total weights for value $v_p$ is at least $\rho$. Therefore, $c_q \leq 1$ and $v_q = v_p$.
\end{proof}
\end{claim*}


\begin{claim*}
$\forall$ non-faulty $p$ and $q$, if $c_p = c_q = 1$, then $v_p = v_q$. 
\begin{proof}
$c_p = c_q = 1$ implies that leader must be faulty, otherwise $c_p = c_q = 2$. Also, the total weights node $p$ and $q$ gathered for $v_p$ and $v_q$ both lie between $1 - \rho$ and $\rho$, which implies that at least one non-faulty node in round 3 got majority weights $\geq 1 - \rho$ for value $v_p$ and $v_q$, respectively. After round 2,  each non-faulty process holds $1 - \rho$ of values from all non-faulty process, so they have $1 - \rho$ values which are exactly the same. For a non-faulty node to send out a value in round 3, this node must have at least $1 - \rho$ weights for that value. So there is at least $1 - 2*\rho$ for $v_p$ and $v_q$ among the non-faulty nodes. Suppose $v_q != v_p$. since $\rho < 1/3$, $1 - 2*\rho + 1 - 2*\rho = 2 - 4*\rho > 1 - \rho$, which gives a contradiction. Therefore, $v_p = v_q$.
\end{proof}
\end{claim*}

\begin{t1}
WEIGHTED-GRADECAST satisfy all Gradecast properties
\begin{proof}
Property 1 can be directly induced from the algorithm. If the leader is non-faulty with value $v$,  then after the first two rounds, every non-faulty node has at least $1 - \rho$ of $v$. So in Phase 3, every non-faulty node will send $v$ to all other nodes, then after Phase 3 every non-faulty node gets at least $1 - \rho$ of $v$. Thus, any non-fault node $p$ will have $v_p = v$ and $c_p = 2$, which implies property 1. 
 
Property 3 can be obtained from claim 1. Claim 1 is actually equivalent to arguing that there couldn't be the case that one non-fault node has confidence 2 for the leader while another has confidence 0. That is to say, for all non-faulty $q$ and $q'$, $|c_q - c_q'| \leq 1$, which proves property 3 of Gradecast.

For property 2, we have the following two cases. 

Case 1: $c_q = 2$ or $c_q' = 2$. Without loss of generality,  we assume $c_q = 2$. From claim 1, we know that $c_q' \leq 1$ and $v_q = v_q'$,  which satisfy property 2.

Case 2: $c_q = c_q' = 1$. From claim 2, we have $v_q = v_q'$. So, this case also satisfy property 2. 
Thus, for every non-faulty nodes $q$, $q'$: if $c_q > 0$ and $c_q' > 0$, then $v_q = v_q'$. Property 2 holds.

Therefore, \textit{WEIGHTED-GRADECAST} satisfy all Gradecast properties. 
\end{proof}
\end{t1}


\begin{l1}
(Persistence of Agreement): Assuming $\rho < 1/3$, if all correct processes  
prefer a same value $v$ at the beginning of a round, then all non-faulty nodes that are still in the main loop exit the loop and update their value to $v$. 
\begin{proof}
All non-faulty nodes see at least $1 - \rho$ of $v$ with confidence 2. Thus by Line 5, 7 they update their value to $v$, and (if they are still in the loop) by Line 9, 10 they all exit the loop.
\end{proof}
\end{l1}


Fig. 2 gives the BYZCONSENSUS-EARLY algorithm to solve the WBA problem based on WEIGHTED-GRADECAST with early stopping property. 

Denote by $\cup BAD_r$ the union of all $BAD$ variables for non-faulty nodes at the beginning of round r. Similarly, denote by $\cap BAD_r$ the intersection of all $BAD$ variables for non-faulty nodes.
at the beginning of iteration r.

\begin{l1}
(Early Stopping): if there are only $\rho' \leq \rho$ failures,  BYZCONSENSUS-EARLY terminates within $min\{\alpha_\rho' + 1, \alpha_\rho\}$ rounds. 
\begin{proof}
\end{proof}
\end{l1}

\begin{t1}
BYZCONSENSUS-EARLY solves the WBA problem.
\begin{proof}
From Lemma 1 it is clear that validity holds. To show that agreement holds, we need to consider the following two cases. 

Case 1: if a non-faulty node passes the condition of Line 9 in the first t rounds, then by claim 7 and Lemma 1 agreement holds.  

Case 2: if no non-faulty node passes the condition of Line 9 in the first $t$ rounds, then all non-faulty nodes perform the main loop of BYZCONSENSUS $t + 1$ times. By claim 6 and Lemma 1 this means that in every round of the first $t$ rounds there is some pair of non-faulty nodes that have different values of $maj$. By Claim 5, $|\cap BAD_{t+1}| > | \cap BAD_t| > ... > |\cap BAD_1| = 0$. Thus, $|\cap BAD_{t+1}| \geq t$. Thus, in iteration $t + 1$ all non-faulty nodes ignore messages from all Byzantine nodes. Therefore, all non-faulty nodes see only $1 - \rho$ of same set of values  and thus they agree on the value of $maj$. By Claim 6 all non-faulty nodes end round $t + 1$ with the same value of $v$
\end{proof}
\end{t1}

\begin{algorithm}
\caption{WEITED-GRADECAST ($q$, IGNORE{$p$})}
\begin{algorithmic}[1]
\STATE \textbf{set} ignore all messages being received below from nodes in IGNORE\textit{p};
\IF {$\textit{p} = \textit{q}$} 
\STATE $v$ = \lq{the input value}\rq
\ENDIF
\item[]
\item[]
\textit{/* Dissemination */}
\STATE \textbf{Phase 1:} The leader $q$ sends v to all:
\STATE \textbf{Phase 2:} \textit{p} sends the value received from \textit{q} to all;
\item[]
\item[]
\textit{/* Notations */}
\STATE \textbf{let} $<$\textit{j}, \textit{${v}_{j}$}$>$ represent that \textit{p} received \textit{${v}_{j}$} from \textit{j};
\STATE \textbf{let} \textit{maj} be a value received the most among such values;
\STATE \textbf{let} $w$ be the total weights gathered for $maj$;
\item[]
\item[]
\textit{/* Support */}
\STATE \textbf{Phase 3:} \IF {$w \geq 1 - \rho $} 
\STATE $p$ sends \textit{maj} to all;
\ENDIF
\item[]
\item[]
\textit{/* Notations */}
\STATE \textbf{let} $<$\textit{j}, \textit{${v'}_{j}$}$>$ represent that \textit{p} received \textit{${v'}_{j}$} from \textit{j};
\STATE \textbf{let} \textit{maj'} be a value received the most among such values;
\STATE \textbf{let} $w'$ be the total weights gathered for \textit{maj'};
\item[]
\item[]
\textit{/* Grading */}
\STATE \textbf{if} $w' \geq 1 - \rho $  \textbf{set} \textit{${v}_{p}$} := \textit{maj'} and \textit{${c}_{p}$} := 2;
\STATE \textbf{otherwise}, \textbf{if} $w' > \rho $ \textbf{set} \textit{${v}_{p}$} := \textit{maj'} and \textit{${c}_{p}$} := 1;
\STATE  \textbf{otherwise} \textbf{set} $\textit{${v}_{p}$} = \perp$ and \textit{${c}_{p}$} = 0;
\item[]
\STATE \textbf{return} $<$\textit{q},\textit{${v}_{p}$},\textit{${c}_{p}$}$>$
\end{algorithmic}
\end{algorithm}

\begin{algorithm}
\caption{BYZCONSENSUS-EARLY}\label{euclid}
\textit{/* Initialization on node p */}
\begin{algorithmic}[1]
\STATE $\textbf{set} ~\textit{BAD} := \emptyset;$
\item[]
\item[]
\textit{/* Main loop */}
\FOR{$\textit{r} := 1$ to $\textit{t} + 1$}
\STATE \textbf{WEIGHTED-GRADECAST}(\textit{p}, \textit{BAD}) with input value \textit{v}
\item[]
\item[]
\textit{/* Notations */}
\STATE \textbf{let} $<$\textit{q}, \textit{v}, \textit{c}$>$ represent that \textit{q} gradecasted \textit{v} with confidence \textit{c}
\STATE \textbf{let} \textit{maj} be the value received the most among values with confidence $\geq$ 1
(if there is more than one such value, take the lowest)
\STATE \textbf{let} $w$ be the total weights of \textit{maj} with confidence 2
\item[]
\item[]
\textit{/* Updates */}
\STATE \textbf{set} \textit{v} $:=$\textit{maj}
\STATE \textbf{set} \textit{BAD} $:=$ \textit{BAD} $\cup$ \{$q \mid$ received $<q$, $\ast$, $c>$ with $c \leq$ 1\}
\IF {$w \geq 1 - \rho$}
\STATE break loop
\ENDIF
\ENDFOR
\item[]
\STATE \textbf{if} executed for $<$ t + 1 iterations then participate in one more iteration
\STATE return \textit{v}

\end{algorithmic}
\end{algorithm}

\begin{algorithm}
\caption{BYZCONSENSUS-ANCHOR}\label{euclid}
\textit{/* Initialization on node $p_i$ */}
\begin{algorithmic}[1]
\STATE $\textbf{set} ~\textit{BAD} := \emptyset;$
\item[]
\item[]
\textit{/* Main loop */}
\FOR{$\textit{r} := 1$ to $\alpha_\rho$}
\STATE \textbf{WEIGHTED-GRADECAST}(\textit{p}, \textit{BAD}) with input value \textit{v}
\item[]
\item[]
\textit{/* Notations */}
\STATE \textbf{let} $<$\textit{q}, \textit{v}, \textit{c}$>$ represent that \textit{q} gradecasted \textit{v} with confidence \textit{c}
\STATE \textbf{let} \textit{maj} be the value received the most among values with confidence $\geq$ 1
(if there is more than one such value, take the lowest)
\STATE \textbf{let} $w$ be the total weights of \textit{maj} with confidence 2
\item[]
\item[]
\textit{/* Updates */}
\STATE \textbf{set} \textit{v} $:=$\textit{maj}
\STATE \textbf{set} \textit{BAD} $:=$ \textit{BAD} $\cup$ \{$q \mid$ received $<q$, $\ast$, $c>$ with $c \leq$ 1\}
\item[]
\item[]
\textit{/* King Phase */}
\IF {$r = i$} 
\STATE send $v$ to all other process
\ENDIF
\STATE receive $kingvalue$ from $p_r$
\IF {$w < 1 - \rho$}
\STATE $v = kingvalue$ 
\ENDIF
\ENDFOR
\item[]
\STATE return \textit{v}

\end{algorithmic}
\end{algorithm}

\begin{l1} 
Assuming $\rho < 1/3$, BYZCONSENSUS-ANCHOR algorithm satisfies persistence of agreement.
\begin{proof}
If all correct processes agree on value $v$ at the beginning of a round, after executing WEITED-GRADECAST, they all see at least $1 - \rho$ of $v$ with confidence 2. Thus,  by line  5 and 7 they all update their value to be $v$. By line 9 and 10, they will choose $v$ instead of $kingvalue$. So all correct process agree on value $v$ at next round.  
\end{proof}
\end{l1}

\begin{l1}
There is at least one round in which the king is correct.
\begin{proof}
By assumption, the total weight of processes that
have failed is less than ρ. The for loop is executed αρ times.
By definition of $\alpha_\rho$, there exists at least one round in which
the king is correct.
\end{proof}
\end{l1}

\begin{t1}
BYZCONSENSUS-ANCHOR solves the weighted agreement problem in $\alpha_\rho$ rounds for $\rho < 1/3$.
\begin{proof}
From lemma 2, validity holds. Since the algorithm only runs $\alpha_\rho$ rounds, termination is guaranteed. From Lemma 3, there is at least one round the king is correct. In that round, every correct process will choose either king's value or its own value. One process can choose its own value only when it has at least $1 - \rho$ weights for its own value. From Claim 3,  we know that if one correct process chooses its own value $v$,  then all other process set their value to be $v$ by Line 7. So every correct process has the same value. So agreement property is satisfied. 
\end{proof}
\end{t1}

\section{Updating Weights}
In this section, the case when the system is required to solve
BA multiple times is considered. This case arises in most real-life applications of BA, such as, maintenance of replicated data
and fault-tolerant file systems [20]. In addition, each execution
of the BA protocol provides certain feedback in terms of the
processes’ behaviour. Every process stores a set of processes it believes to be faulty, which motivates us to devise an algorithm to achieve consensus on the faulty set among all non-faulty processes. Fig. 3 gives an algorithm to detect the set of processes that are definitely faulty and let all non-faulty processes agree on this set. After having agreed on the faulty set, weights can be updated by setting the weights of faulty processes to be zero and giving non-faulty ones higher weights. We can notice that for the BYZCONSENSUS-EARLY algorithm, there is no need to update weights since the number of rounds does not depend on the weights of process. However, the number of rounds in BYZCONSENSUS-ANCHOR algorithm depends on the anchor of the system. If reliable processes has higher weights, then the anchor would be reduced. Therefore, the WEIGHT-UPDATE algorithm should be combined with BYZCONSENSUS-ANCHOR in practical settings.

\begin{algorithm}
\caption{WEIGHT-UPDATE}\label{euclid}
\textit{/* Executed on process $p_{i}$ with faulty set $BAD_{i} $ */}

\begin{algorithmic}[1]
\STATE $\textbf{set} ~\textit{faulty} := \emptyset;$
\item[]
\item[]
\textit{/* Detect Faulty */}
\FOR {$\textit{k} := 1$ to $\textit{n}$}
\IF {$k \in BAD_i$}
\STATE $value$ := BYZCONSENSUS-EARLY(1)
\ELSE
\STATE $value$ := BYZCONSENSUS-EARLY(0)
\ENDIF
\IF {$value = 1$} 
\STATE $faulty := faulty \cup {k}$
\ENDIF
\ENDFOR
\item[]
\item[]
\textit{/* Update Weights */}
\STATE float $totalWeight := 1.0$
\FORALL {$j \in faulty$}
\STATE $totalWeight := totalWeight - w[j]$
\STATE $w[j] := 0$
\ENDFOR
\FORALL {$j$}
\STATE $w[j] := w[j]/totalWeight$
\ENDFOR
\end{algorithmic}
\end{algorithm}

The correctness of the algorithm in Fig.4 is shown in the following lemma and theorem. Let $r_i$ be the rounds process $p_i$ took to exit the main loop in BYZCONSENSUS-EARLY. Let $r_m$ be the minimum round among $r_i$  

\begin{l1}
WEIGHT-UPDATE guarantees that $r_m - 1$ faulty processes will be detected. 
\begin{proof}

\end{proof}
\end{l1}


\section{Simulation}

\section{Conclusion}
This paper has presented a weighted version of the Byzantine Agreement Problem and provided solutions for the problem in a synchronous distributed system. We show that the
weighted version has the advantage of using fewer messages
and tolerating more failures (under certain conditions) than
is required by the lower bound for the unweighted version.
These algorithms have applications in many systems in which
there are two classes of processes: trusted and untrusted
processes. Instead of tolerating any f faults in the BA problem,
these algorithms tolerate failure of processes with total weight
less than f=N. For example, an implementation can now
tolerate more than f faults of untrusted processes; but, fewer
than f faults of trusted processes depending on the weight
assignment. A fault-tolerant method has also been presented
to update the weights at all the correct processes. This algorithm is useful for many applications where the agreement is
required multiple times. Our update algorithm guarantees that
the weight of a correct process is never reduced and the weight
of any faulty process, suspected by correct processes whose
total weight is at least 1=4, is reduced to 0.
\section*{References}

Please number citations consecutively within brackets \cite{b1}. The 
sentence punctuation follows the bracket \cite{b2}. Refer simply to the reference 
number, as in \cite{b3}---do not use ``Ref. \cite{b3}'' or ``reference \cite{b3}'' except at 
the beginning of a sentence: ``Reference \cite{b3} was the first $\ldots$''

Number footnotes separately in superscripts. Place the actual footnote at 
the bottom of the column in which it was cited. Do not put footnotes in the 
abstract or reference list. Use letters for table footnotes.

Unless there are six authors or more give all authors' names; do not use 
``et al.''. Papers that have not been published, even if they have been 
submitted for publication, should be cited as ``unpublished'' \cite{b4}. Papers 
that have been accepted for publication should be cited as ``in press'' \cite{b5}. 
Capitalize only the first word in a paper title, except for proper nouns and 
element symbols.

For papers published in translation journals, please give the English 
citation first, followed by the original foreign-language citation \cite{b6}.

\begin{thebibliography}{00}
\bibitem{b1} G. Eason, B. Noble, and I. N. Sneddon, ``On certain integrals of Lipschitz-Hankel type involving products of Bessel functions,'' Phil. Trans. Roy. Soc. London, vol. A247, pp. 529--551, April 1955.
\bibitem{b2} J. Clerk Maxwell, A Treatise on Electricity and Magnetism, 3rd ed., vol. 2. Oxford: Clarendon, 1892, pp.68--73.
\bibitem{b3} I. S. Jacobs and C. P. Bean, ``Fine particles, thin films and exchange anisotropy,'' in Magnetism, vol. III, G. T. Rado and H. Suhl, Eds. New York: Academic, 1963, pp. 271--350.
\bibitem{b4} K. Elissa, ``Title of paper if known,'' unpublished.
\bibitem{b5} R. Nicole, ``Title of paper with only first word capitalized,'' J. Name Stand. Abbrev., in press.
\bibitem{b7} Y. Yorozu, M. Hirano, K. Oka, and Y. Tagawa, ``Electron spectroscopy studies on magneto-optical media and plastic substrate interface,'' IEEE Transl. J. Magn. Japan, vol. 2, pp. 740--741, August 1987 [Digests 9th Annual Conf. Magnetics Japan, p. 301, 1982].
\bibitem{b7} M. Young, The Technical Writer's Handbook. Mill Valley, CA: University Science, 1989.
\end{thebibliography}

[7]Paul Feldman and Silvio Micali. Optimal algorithms for byzantine agreement. In STOC ’88:
Proceedings of the twentieth annual ACM symposium on Theory of computing, pages 148–161,
New York, NY, USA, 1988. ACM.


[8]Simple Gradecast Based Algorithm

\begin{appendices}
\section{Claims}
\begin{claim*}
If $\cup BAD_r$ contains only faulty nodes, then the properties of gradecast, following its execution in Line 3, hold.
\begin{proof}
If ∪BADr contains only faulty nodes, then the properties of gradecast, following its execution in Line 3, hold
\end{proof}
\end{claim*}

\begin{claim*}
If $\cup BAD_r$ contains only faulty nodes, then $\cup BAD_{r+1}$ contains only faulty nodes.
\begin{proof}
Consider a non-faulty node q. By Claim 2, q’s gradecast confidence is 2 at all non-faulty
nodes. Thus, no non-faulty node adds q to BAD in the current iteration. Therefore, ∪BADr+1
contains only faulty nodes.
\end{proof}
\end{claim*}

\begin{claim*}
If non-faulty nodes $p$, $q$ have different values of $maj$ at round $r$, then $|\cap BAD_{r+1}| > | \cap BAD_r|$.
\begin{proof}
If p has a different value of maj than q, then (w.l.o.g.) by the definition of maj (Line 5)
there is some Byzantine node z such that p received hz, u, ∗i from z’s gradecast, and q received
hz, u′, ∗i, s.t. u 6= u′. By the properties of gradecast, all non-faulty nodes have confidence of at
most 1 for z’s gradecast. Therefore, by Line 8, all non-faulty nodes add z to BAD. That is,
z ∈ ∩BADr+1.
To conclude the proof, we need to show that z / ∈ ∩BADr. Since p and q see different confidence
for z’s gradecast, we conclude that some non-faulty node didn’t ignore z’s messages. (Otherwise, by
Claim 1, z gradecast confidence would have been 0 at all non-faulty nodes.) Therefore, we conclude
that z / ∈ ∩BADr
\end{proof}
\end{claim*}

\begin{claim*}
If all non-faulty nodes have the same value of $maj$ at round $r$, then all non-faulty nodes end round $r$ with the same value $v$.
\begin{proof}
Immediate from Line 7.
\end{proof}
\end{claim*}

\begin{claim*}
If some node $p$ breaks the main loop due to Line 9 during round $r$, then all non-faulty nodes end round $r$ with the same value $v$.

\begin{proof}
For p to pass the condition of Line 9, maj must be at least n − t. That is, p sees at
least n − t gradecast values equal to maj with confidence 2. From the properties of gradecast, all
other non-faulty nodes see n − t gradecast values equal to maj with confidence ≥ 1. By Line 5,7
they all update their value to be that same value
\end{proof}

\end{claim*}

\end{appendices}

\end{document}
