\documentclass[conference]{IEEEtran}
\IEEEoverridecommandlockouts
% The preceding line is only needed to identify funding in the first footnote. If that is unneeded, please comment it out.
\usepackage{cite}
\usepackage{amsmath,amssymb,amsfonts}
\usepackage{graphicx}
\usepackage{textcomp}
\usepackage{algorithm}
\usepackage[]{algorithmic}
\usepackage{amsthm}
\newtheorem{t1}{Theorem}
\newtheorem{l1}{Lemma}
\newtheorem{claim*}{Claim}

%\usepackage{algpseudocode}
\def\BibTeX{{\rm B\kern-.05em{\sc i\kern-.025em b}\kern-.08em
    T\kern-.1667em\lower.7ex\hbox{E}\kern-.125emX}}
\begin{document}
\pagenumbering{roman}
\title{Gradecast for Weighted Byzantine Agreement\\
}

\author{\IEEEauthorblockN{Xiong Zheng}
\IEEEauthorblockA{\textit{Department of Electrical and Computer Engineering} \\
\textit{The University of Texas at Austin}\\
Austin, USA \\
zhengxiongtym@utexas.edu}
\and
\IEEEauthorblockN{Haowei Sun}
\IEEEauthorblockA{\textit{Department of Electrical and Computer Engineering} \\
\textit{The University of Texas at Austin}\\
Austin, USA \\
michelle.shw@utexas.edu}
}

\maketitle

\begin{abstract}
Gradecast is simple three-round algorithm presented by Feldman and Micali. This paper presents a weighted version of Gradecast based on which two algorithms are given to solve multi-value weighted Byzantine agreement problem. In the weighted Byzantine agreement problem, each machines each machine is assigned a weight which indicates its reliability. Instead of assuming that at most $f$ out of $N$ machines can fail, the algorithms assume that the total
weight of the machines that fail is at most $f/N$. The first algorithm has early stopping property: a) if all correct process prefer same value then algorithm terminates in two rounds; b) number of rounds depends on number of actual failures instead of the number of failures the system needs to tolerate. The second algorithm is can terminates in rounds equal to the \textit{anchor} of the system, which is defined as the least number of processes whose total weight exceeds the weight the system can tolerate. Also, a weight update algorithm is given to update weights of processes after each execution of the weighted Byzantine Agreement. This weight update algorithm provides guarantees on the number of faulty processes it can detect at each invocation. 
\end{abstract}

\begin{IEEEkeywords}
Gradecast, Weighted Byzantine Agreement, Early Stopping
\end{IEEEkeywords}

\section{Introduction}
The Byzantine Agreement (BA) [1], [2] is a fundamental
problem in distributed computing with extensive literature [3]–
[9]. In the usual set-up, there are N processes required to agree
on a common value, given that at most f of them may show
arbitrary or Byzantine behavior. In real-life applications, there
may be multiple classes of processes. For example, in a mobile
computing scenario, mobile hosts may be more likely to fail
compared to mobile stations. In another example, processes in
the same data center may be more likely to fail together. This
paper defines a weighted version of the Byzantine Agreement
Problem (WBA) and provides lower bounds and algorithms
for it. In WBA, each process Pi is assigned a weight w[i],
such that 0 ≤ w[i] ≤ 1 and the sum of all weights is 1.
The WBA problem requires a protocol to reach consensus
when the total weight of the failed processes is at most ρ.
The weighted version gives some surprising results for the
BA problem. First, even if greater than N=3 processes are
Byzantine, the system can still reach consensus so long as ρ
is less than 1=3. This result is quite useful in the system with
a small set of trusted processes and a large set of less trusted
processes.

Secondly, the message complexity and the number of rounds
required to achieve consensus for the weighted version is
shown to always be less than or equal to those for the unweighted version. Suppose the system must tolerate ρ = f=N
for any integer f such that 0 ≤ f < N=3. It is known 

The organization of the rest of this paper is as follows.
First, the model of the system that runs algorithms is defined.
Second, the Weighted-Queen Algorithm is given. Then, the
Weighted-King Algorithm is discussed. Next, a simple weight
update method is given. Then, some initial weight assignment
strategies are presented. Finally, concluding remarks are made

\section{System Model}
The system model used in this paper is a distributed system
with $n$ processes, $p_1...p_n$, with a completely connected topology. The underlying system is assumed to be synchronous; i.e.,
there is an upper bound on the message delay and on the duration of actions performed by processes. The model assumes
that processes may fail; but, the underlying communication
system is reliable and satisfies first-in first-out (FIFO) message
ordering. The processes may fail in an arbitrary fashion; in
particular, they may lie and collude with other failed processes
to foil any protocol. The processes that do not fail in any
computation are called correct processes. Assume that there
is a non-negative weight $w[i]$ associated with each process
$P_i$. All processes in the system have complete knowledge
of weights of all the processes. For simplicity, assume that
weights are normalized; i.e., the sum of all weights is one.
Let $ρ$ be the sum of weights of all failed processes. This paper
assumes that $ρ$ is strictly less than one.The Weighted Byzantine Agreement (WBA) problem can be specified as follows. All processes propose a binary value with the goal of deciding on one common value. Given a
weight assignment to all processes, and the assumption that
the weight of processes that fail during the execution is at most
$ρ$, the WBA problem is to design a protocol that satisfies the
following conditions:

• \textbf{Agreement}: Two correct processes cannot decide on
different values.

• \textbf{Validity}: The value decided must be proposed by some
correct process.

• \textbf{Termination}: All correct processes decide in finite number of steps.
The following lower bounds follow easily from the standard
BA lower bound arguments.

For simplicity, also assume that weights associated with
$p_i$ are in non-increasing order. This can be achieved by
renumbering processes, if necessary. Given any $\rho$ and weight
assignment $w$, define the anchor $\alpha_\rho$ as the minimum number
of processes such that the sum of their weights is strictly
greater than $\rho$. The anti version of $\alpha_\rho$ is $\theta_\rho$ which id defined as the maximum number of processes such that the sum of their weights is strictly greater than $\rho$. Formally,
\begin{center}
$\alpha_\rho = min\{k~|~\sum_{i = 1}^k {w[i]} > \rho\}$
\end{center}
\begin{center}
$\theta_\rho = max\{k~ |~ \sum_{i = 1}^k{w[i]} > \rho\}$
\end{center}

The significance of $\alpha_\rho$ is that at least one process
from $p_1...p_\alpha$ is guaranteed to be correct, if assuming there weights associated with $p_i$ are in non-increasing order. When $\rho$ is zero, $\alpha_\rho$ is 1. The largest possible value of $\alpha_\rho$ is $n$, because $\rho$ < 1 by assumption. The significance of $\theta_\rho$ is that at least one process from $p_1...p_{\theta_{\rho}}$ is guaranteed to be correct,  when we don't assume weights are ordered in a non-increasing fashion. 

\section{Weighted Gradecast Based Algorithms}
In this section, several algorithms are given to solve WBA problem based on weighted version of Gradecast. Gradecast[7] is a three-round distributed algorithm that ensures some properties that are similar to those of broadcast. Specifically, in Gradecast there is a sender node $p$ that sends a value $v$ to all other. Each node $q'$s output is a pair $\langle v_q, c_q \rangle$ where $v_q$ is the value $q$ thinks $p$ has sent and $c_q$ is $q'$s confidence in this value. The Gradecast properties ensure that:

1. If $p$ is non-faulty then $v_q$ = $v$ and $c_q = 2$, for every non-faulty $q$;

2. For every non-faulty nodes $q$, $q'$: if $c_q > 0$ and $c_q' > 0$ then $v_q = v_q'$;

3. $|c_q - c_q'| \leq 1$ for every non-faulty nodes $q$, $q'$.


\subsection{WEIGHTED-GRADECAST Algorithm}
For the WBA problem,  in order to apply gradecast, we simply need to do some minor modification of the original gradecast algorithm proposed by Michael et. all in [8]. The algorithm uses constant size messages and requires that $\rho < 1/3$. Each process has a preferred value for each round. The modified algorithm is presented in Fig. 1. To show the WEIGHTED-GRADECAST algorithm can solve the WBA problem, we need to show that the above three Gradecast properties hold and the proof for Byzantine consensus in [8] still holds, which can be shown through the following claims.
 
\begin{claim*}
if non-faulty node $p$ has $c_q = 2$, then any other non-faulty node $q$ has $c_q \geq 1$ and $v_q = v_p$
\begin{proof}
Since $c_p = 2$, from the grading part of \textbf{WEIGHTED-GRADECAST} we know that the total weight $w'$ (for $v_p) \leq 1 - \rho$. There are at most $\rho$ faulty nodes which implies that the weights from non-faulty nodes for value $v_p$ in round 3 is at least $1 - 2*\rho$. Similarly,  the weights from non-faulty Since $\rho < 1/3$, we have $1 - 2*\rho > \rho$. So for any other non-fault node $q$, the total weights for value $v_p$ is at least $\rho$. Therefore, $c_q \leq 1$ and $v_q = v_p$.
\end{proof}
\end{claim*}


\begin{claim*}
$\forall$ non-faulty $p$ and $q$, if $c_p = c_q = 1$, then $v_p = v_q$. 
\begin{proof}
$c_p = c_q = 1$ implies that leader must be faulty, otherwise $c_p = c_q = 2$. Also, the total weights node $p$ and $q$ gathered for $v_p$ and $v_q$ both lie between $1 - \rho$ and $\rho$, which implies that at least one non-faulty node in round 3 got majority weights $\geq 1 - \rho$ for value $v_p$ and $v_q$, respectively. After round 2,  each non-faulty process holds $1 - \rho$ of values from all non-faulty process, so they have $1 - \rho$ values which are exactly the same. For a non-faulty node to send out a value in round 3, this node must have at least $1 - \rho$ weights for that value. So there is at least $1 - 2*\rho$ for $v_p$ and $v_q$ among the non-faulty nodes. Suppose $v_q != v_p$. since $\rho < 1/3$, $1 - 2*\rho + 1 - 2*\rho = 2 - 4*\rho > 1 - \rho$, which gives a contradiction. Therefore, $v_p = v_q$.
\end{proof}
\end{claim*}

\begin{t1}
WEIGHTED-GRADECAST satisfy all Gradecast properties
\begin{proof}
Property 1 can be directly induced from the algorithm. If the leader is non-faulty with value $v$,  then after the first two rounds, every non-faulty node has at least $1 - \rho$ of $v$. So in Phase 3, every non-faulty node will send $v$ to all other nodes, then after Phase 3 every non-faulty node gets at least $1 - \rho$ of $v$. Thus, any non-fault node $p$ will have $v_p = v$ and $c_p = 2$, which implies property 1. 
 
Property 3 can be obtained from claim 1. Claim 1 is actually equivalent to arguing that there couldn't be the case that one non-fault node has confidence 2 for the leader while another has confidence 0. That is to say, for all non-faulty $q$ and $q'$, $|c_q - c_q'| \leq 1$, which proves property 3 of Gradecast.

For property 2, we have the following two cases. 

Case 1: $c_q = 2$ or $c_q' = 2$. Without loss of generality,  we assume $c_q = 2$. From claim 1, we know that $c_q' \leq 1$ and $v_q = v_q'$,  which satisfy property 2.

Case 2: $c_q = c_q' = 1$. From claim 2, we have $v_q = v_q'$. So, this case also satisfy property 2. 
Thus, for every non-faulty nodes $q$, $q'$: if $c_q > 0$ and $c_q' > 0$, then $v_q = v_q'$. Property 2 holds.

Therefore, \textit{WEIGHTED-GRADECAST} satisfy all Gradecast properties. 
\end{proof}
\end{t1}

\begin{algorithm}
\caption{WEIGHTED-GRADECAST ($q$, IGNORE{$p$})}
\begin{algorithmic}[1]
\STATE \textbf{set} ignore all messages being received below from nodes in IGNORE\textit{p};
\IF {$\textit{p} = \textit{q}$} 
\STATE $v$ = \lq{the input value}\rq
\ENDIF
\item[]
\item[]
\textit{/* Dissemination */}
\STATE \textbf{Phase 1:} The leader $q$ sends v to all:
\STATE \textbf{Phase 2:} \textit{p} sends the value received from \textit{q} to all;
\item[]
\item[]
\textit{/* Notations */}
\STATE \textbf{let} $<$\textit{j}, \textit{${v}_{j}$}$>$ represent that \textit{p} received \textit{${v}_{j}$} from \textit{j};
\STATE \textbf{let} \textit{maj} be a value received the most among such values;
\STATE \textbf{let} $w$ be the total weights gathered for $maj$;
\item[]
\item[]
\textit{/* Support */}
\STATE \textbf{Phase 3:} \IF {$w \geq 1 - \rho $} 
\STATE $p$ sends \textit{maj} to all;
\ENDIF
\item[]
\item[]
\textit{/* Notations */}
\STATE \textbf{let} $<$\textit{j}, \textit{${v'}_{j}$}$>$ represent that \textit{p} received \textit{${v'}_{j}$} from \textit{j};
\STATE \textbf{let} \textit{maj'} be a value received the most among such values;
\STATE \textbf{let} $w'$ be the total weights gathered for \textit{maj'};
\item[]
\item[]
\textit{/* Grading */}
\STATE \textbf{if} $w' \geq 1 - \rho $  \textbf{set} \textit{${v}_{p}$} := \textit{maj'} and \textit{${c}_{p}$} := 2;
\STATE \textbf{otherwise}, \textbf{if} $w' > \rho $ \textbf{set} \textit{${v}_{p}$} := \textit{maj'} and \textit{${c}_{p}$} := 1;
\STATE  \textbf{otherwise} \textbf{set} $\textit{${v}_{p}$} = \perp$ and \textit{${c}_{p}$} = 0;
\item[]
\STATE \textbf{return} $<$\textit{q},\textit{${v}_{p}$},\textit{${c}_{p}$}$>$
\end{algorithmic}
\end{algorithm}

\subsection{BYZCONSENSUS-EARLY Algorithm}
In this part, an algorithm is given which takes $\theta_\rho$ rounds in the worst case , each round of three phases, to solve the WBA problem.  
The BYZCONSENSUS-EARLY algorithm in Fig.2 is based on the unweighted version Byzantine algorithm of Michael, Danny and Ezra [8]. The original GRADECAST is replaced with WEIGHTED-GRADECAST. The basic idea is essentially the same.
WEIGHTED-GRADECAST is applied as a means of forcing the Byzantine nodes to “lie” at the expense of being expelled from the algorithm. That is, At each round, every processes invoke WEIGHTED-GRADECAST to gradecast its own value as the leader and also receive other processes gradecast value . Then after each WEIGHTED-GRADECAST, each process has a pair of gradecast value and confidence for each process. Each process will check each process's value and confidence pair and decide whether to trust it in next round or not based on the following rules: a)  The process gradecasted a value with confidence ≤ 1 will be marked as faulty, and will be ignored for the rest of the algorithm; b) Any value with confidence ≥ 1 will be considered, and $p$ will update its own value to be the majority of values with confidence ≥ 1. Moreover, this mechanism ensures that for a faulty node $z$, if different non-faulty nodes consider different values for $z$'s gradecast, then at least one of them should obtain the value with zero confidence. For example, one considers
$z$ gradecasted “0” with confidence 1, and the other considers $z$ gradecast’s confidence to be 0. The result of such a case is that all non-faulty nodes will mark $z$ to be faulty, and will remove it from
the algorithm. In other words, a Byzantine node can produce, using WEIGHTED-GRADECASE, contradicting values to non-faulty nodes at most once. 


Denote by $\cup BAD_r$ the union of all $BAD$ variables for non-faulty nodes at the beginning of round r. Similarly, denote by $\cap BAD_r$ the intersection of all $BAD$ variables for non-faulty nodes.
at the beginning of iteration $r$. Claim 3 to 8 are from [8], so we simply list here without proof. 

\begin{claim*}
If $\cup BAD_r$ contains only faulty nodes, then the properties of gradecast, following its execution in Line 3, hold.
\end{claim*}

\begin{claim*}
If $\cup BAD_r$ contains only faulty nodes, then $\cup BAD_{r+1}$ contains only faulty nodes.
\end{claim*}

\begin{claim*}
$\cup BAD_r$ never contains non-faulty processes.
\end{claim*}


\begin{claim*}
If non-faulty nodes $p$, $q$ have different values of $maj$ at round $r$, then $|\cap BAD_{r+1}| > | \cap BAD_r|$.
\end{claim*}

\begin{claim*}
If all non-faulty nodes have the same value of $maj$ at round $r$, then all non-faulty nodes end round $r$ with the same value $v$.
\end{claim*}

\begin{claim*}
If some node $p$ breaks the main loop due to Line 9 during round $r$, then all non-faulty nodes end round $r$ with the same value $v$.
\end{claim*}

\begin{l1}
(Persistence of Agreement): Assuming $\rho < 1/3$, if all correct processes  
prefer a same value $v$ at the beginning of a round, then all non-faulty nodes that are still in the main loop exit the loop and update their value to $v$. 
\begin{proof}
All non-faulty nodes see at least $1 - \rho$ of $v$ with confidence 2. Thus by Line 5, 7 they update their value to $v$, and (if they are still in the loop) by Line 9, 10 they all exit the loop.
\end{proof}
\end{l1}

\begin{l1}
BYZCONSENSUS-EARLY satisfy early stopping property, which includes the following two aspects. 

1. If all correct processes prefer the same value at the beginning of a round,  then the algorithm terminates after two more rounds.

2. If there are only $\rho' \leq \rho$ failures,  BYZCONSENSUS-EARLY terminates within $min\{\alpha_{\rho'} + 1, \alpha_\rho\}$ rounds. 
\begin{proof}
First, if all correct processes prefer the same value $v$ at the beginning of a round, then from Lemma 1 every correct processes exit the main loop of BYZCONSENSUS-EARLY, update their value to $v$ and participate one more round. Therefore, every correct process terminates in two rounds if they prefer same value. 
Second, if there are only $\rho' \leq \rho$ failures. By Claim 5, $|\cap BAD_{\theta_{\rho'}}| > | \cap BAD_{\theta_{\rho'} - 1}| > ... > |\cap BAD_1| = 0$. Thus, $|\cap BAD_{\theta_{\rho'}}| \geq \theta_{\rho'} - 1$. By definition of $\theta_{\rho'}$ we know that in iteration $\theta_{\rho'}$ all non-faulty nodes ignore messages from all Byzantine nodes. Therefore, all non-faulty nodes see only $1 - {\rho'}$ of same set of values which is greater than $1 - \rho$ and thus they agree on the value of $maj$. By Lemma 1, in the worst cast every correct process exit main loop in round $\theta_{\rho'}$. According to Line 13, the algorithm terminates within $min\{\theta_{\rho'} + 1, \theta_\rho\}$ rounds.
\end{proof}
\end{l1}

Compared with WEIGHTED-QUEEN and WEIGHTED-KING algorithm proposed in [6], the first aspect of early stopping is a significant advantage. These two algorithms wouldn't terminate even if all correct process prefer same value at the beginning of a round. Each process is enforced to execute $\alpha_\rho$ rounds in all situations. For the second aspect, when the actual number of failures is significantly smaller than the number of failures we can tolerate, BYZCONSENSUS-EARLY terminates with fewer rounds. However, we cannot ignore the bad side of BYZCONSENSUS-EARLY algorithm. In the worst case, according to the definition of $\theta_\rho$, when the Byzantine nodes are those with small weights, it can actually take more rounds than the algorithm without weight assignment. 

\begin{t1}
BYZCONSENSUS-EARLY solves the WBA problem.
\begin{proof}
From Lemma 1 it is clear that validity holds. To show that agreement holds, we need to consider the following two cases. 

Case 1: if a non-faulty node passes the condition of Line 9 in the first $\theta_\rho - 1$ rounds, then by claim 7 and Lemma 1 agreement holds.  

Case 2: if no non-faulty node passes the condition of Line 9 in the first $\theta_\rho - 1$ rounds, then all non-faulty nodes perform the main loop of BYZCONSENSUS $\theta_\rho$ times. By claim 6 and Lemma 1 this means that in every round of the first $\theta_\rho - 1$ rounds there is some pair of non-faulty nodes that have different values of $maj$. By Claim 5, $|\cap BAD_{\theta_\rho}| > | \cap BAD_{\theta_\rho - 1}| > ... > |\cap BAD_1| = 0$. Thus, $|\cap BAD_{\theta_\rho}| \geq \theta_\rho - 1$. By definition of $\theta_\rho$ we know that in iteration $\theta_\rho$ all non-faulty nodes ignore messages from all Byzantine nodes. Therefore, all non-faulty nodes see only $1 - \rho$ of same set of values  and thus they agree on the value of $maj$. By Claim 6 all non-faulty nodes end round $\theta_\rho$ with the same value of $v$
\end{proof}
\end{t1}



\begin{algorithm}
\caption{BYZCONSENSUS-EARLY}\label{euclid}
\textit{/* Initialization on node p */}
\begin{algorithmic}[1]
\STATE $\textbf{set} ~\textit{BAD} := \emptyset;$
\item[]
\item[]
\textit{/* Main loop */}
\FOR{$\textit{r} := 1$ to $\theta_\rho$}
\STATE \textbf{WEIGHTED-GRADECAST}(\textit{p}, \textit{BAD}) with input value \textit{v}
\item[]
\item[]
\textit{/* Notations */}
\STATE \textbf{let} $<$\textit{q}, \textit{v}, \textit{c}$>$ represent that \textit{q} gradecasted \textit{v} with confidence \textit{c}
\STATE \textbf{let} \textit{maj} be the value received the most among values with confidence $\geq$ 1
(if there is more than one such value, take the lowest)
\STATE \textbf{let} $w$ be the total weights of \textit{maj} with confidence 2
\item[]
\item[]
\textit{/* Updates */}
\STATE \textbf{set} \textit{v} $:=$\textit{maj}
\STATE \textbf{set} \textit{BAD} $:=$ \textit{BAD} $\cup$ \{$q \mid$ received $<q$, $\ast$, $c>$ with $c \leq$ 1\}
\IF {$w \geq 1 - \rho$}
\STATE break loop
\ENDIF
\ENDFOR
\item[]
\STATE \textbf{if} executed for $<$ t + 1 iterations then participate in one more iteration
\STATE return \textit{v}

\end{algorithmic}
\end{algorithm}

Now, let us look at the message complexity of BYZCONSENSUS-EARLY algorithm. BYZCONSENSUS-EARLY takes $\theta_\rho$ rounds in worse case, each with three phases.  This algorithm takes three phases. In phase one, all processes send their values to all which results in $N^2$ messages. All processes pass messages received from all leaders to all in phase 2, which takes $N^3$ messages. In phase 3, the process which has majority of $1 - \rho$ sends to all which uses $N^2$. Therefore, this algorithm takes  $\theta_\rho*N^3$ messages in worst case. 

\subsection{BYZCONSENSUS-ANCHOR Algorithm}
Another weighted gradecast based algorithm is shown in this part, which terminates in $\alpha_\rho$ rounds with three phases per round. 

\begin{l1} 
Assuming $\rho < 1/3$, BYZCONSENSUS-ANCHOR algorithm satisfies persistence of agreement.
\begin{proof}
If all correct processes agree on value $v$ at the beginning of a round, after executing WEITED-GRADECAST, they all see at least $1 - \rho$ of $v$ with confidence 2. Thus,  by line  5 and 7 they all update their value to be $v$. By line 9 and 10, they will choose $v$ instead of $kingvalue$. So all correct process agree on value $v$ at next round.  
\end{proof}
\end{l1}

\begin{l1}
There is at least one round in which the king is correct.
\begin{proof}
By assumption, the total weight of processes that
have failed is less than ρ. The for loop is executed αρ times.
By definition of $\alpha_\rho$, there exists at least one round in which
the king is correct.
\end{proof}
\end{l1}

\begin{algorithm}
\caption{BYZCONSENSUS-ANCHOR}\label{euclid}
\textit{/* Initialization on node $p_i$ */}
\begin{algorithmic}[1]
\STATE $\textbf{set} ~\textit{BAD} := \emptyset;$
\item[]
\item[]
\textit{/* Main loop */}
\FOR{$\textit{r} := 1$ to $\alpha_\rho$}
\STATE \textbf{WEIGHTED-GRADECAST}(\textit{p}, \textit{BAD}) with input value \textit{v}
\item[]
\item[]
\textit{/* Notations */}
\STATE \textbf{let} $<$\textit{q}, \textit{v}, \textit{c}$>$ represent that \textit{q} gradecasted \textit{v} with confidence \textit{c}
\STATE \textbf{let} \textit{maj} be the value received the most among values with confidence $\geq$ 1
(if there is more than one such value, take the lowest)
\STATE \textbf{let} $w$ be the total weights of \textit{maj} with confidence 2
\item[]
\item[]
\textit{/* Updates */}
\STATE \textbf{set} \textit{v} $:=$\textit{maj}
\STATE \textbf{set} \textit{BAD} $:=$ \textit{BAD} $\cup$ \{$q \mid$ received $<q$, $\ast$, $c>$ with $c \leq$ 1\}
\item[]
\item[]
\textit{/* King Phase */}
\IF {$r = i$} 
\STATE send $v$ to all other process
\ENDIF
\STATE receive $kingvalue$ from $p_r$
\IF {$w < 1 - \rho$}
\STATE $v = kingvalue$ 
\ENDIF
\ENDFOR
\item[]
\STATE return \textit{v}

\end{algorithmic}
\end{algorithm}
\begin{t1}
BYZCONSENSUS-ANCHOR solves the weighted agreement problem for $\rho < 1/3$.
\begin{proof}
From lemma 2, validity holds. Since the algorithm only runs $\alpha_\rho$ rounds, termination is guaranteed. From Lemma 3, there is at least one round the king is correct. In that round, every correct process will choose either king's value or its own value. One process can choose its own value only when it has at least $1 - \rho$ weights for its own value. From Claim 3,  we know that if one correct process chooses its own value $v$,  then all other process set their value to be $v$ by Line 7. So every correct process has the same value. So agreement property is satisfied. 
\end{proof}
\end{t1}

\begin{l1}
$\alpha_{f/N} \leq f + 1$ for all $w$ and $f$.
\begin{proof}
Proved in [6].
\end{proof}
\end{l1}

Let us analyze the message complexity of BYZCONSENSUS-ANCHOR algorithm. BYZCONSENSUS-ANCHOR takes $\alpha_\rho$ rounds in worse case, each with three phases.  This algorithm takes three phases. In phase one, all processes send their values to all which results in $N^2$ messages. All processes pass messages received from all leaders to all in phase 2, which takes $N^3$ messages. In phase 3, the process which has majority of $1 - \rho$ sends to all which uses $N^2$. Therefore, this algorithm takes  $\alpha_\rho*N^3$ messages in worst case. 

Table. 1 gives comparison of different algorithms to solve the WBA problem. Different algorithms have different properties and have different application settings. 

\section{Updating Weights}

In this section, we consider the situation that the system has to do Byzantine agreement multiple times. This situation occurs in lots of practical applications which uses Byzantine agreement,  such as maintenance of replicated data and fault-tolerant file systems [20]. Besides, each process while executing the WBA protocol based on gradecast can detect at least one fault process in each round. However, different processes may have different faulty set, which motivates us to come up with an algorithm to achieve consensus on the faulty set among all non-faulty processes. Fig. 4 shows an algorithm to agree on the faulty set among all non-faulty processes. After having agreed on the faulty set, weights can be updated by setting the weights of faulty processes to be zero and giving non-faulty ones higher weights. Weight updates actually doesn't improve the WEIGHTED-GRADECAST algorithm in terms of message complexity. However,  We can notice that for the BYZCONSENSUS-EARLY and BYZCONSENSUS-ANCHOR algorithm, the number of rounds depend on $\alpha_\rho$ and $\theta_\rho$ respectively, so as the message complexity. These two parameters highly depends on how the weights are assigned. Therefore, weight update does impact the two algorithms. WEIGHT-UPDATE algorithm will increase weight of non-faulty processes. Since if non-faulty processes have higher weights, then we can have smaller $\alpha_\rho$ and $\theta_\rho$. Therefore,  the number of rounds and messages can be reduced.

\begin{algorithm}
\caption{WEIGHT-UPDATE}\label{euclid}
\textit{/* Executed on process $p_{i}$ with faulty set $BAD_{i} $ */}

\begin{algorithmic}[1]
\STATE $\textbf{set} ~\textit{faulty} := \emptyset;$
\item[]
\item[]
\textit{/* Detect Faulty */}
\FOR {$\textit{k} := 1$ to $\textit{n}$}
\IF {$k \in BAD_i$}
\STATE $value$ := BYZCONSENSUS-EARLY(1)
\ELSE
\STATE $value$ := BYZCONSENSUS-EARLY(0)
\ENDIF
\IF {$value = 1$} 
\STATE $faulty := faulty \cup {k}$
\ENDIF
\ENDFOR
\item[]
\item[]
\textit{/* Update Weights */}
\STATE float $totalWeight := 1.0$
\FORALL {$j \in faulty$}
\STATE $totalWeight := totalWeight - w[j]$
\STATE $w[j] := 0$
\ENDFOR
\FORALL {$j$}
\STATE $w[j] := w[j]/totalWeight$
\ENDFOR
\end{algorithmic}
\end{algorithm}


The correctness of the algorithm in Fig.4 is shown in the following lemma and theorem. Let $r_i$ be the rounds process $p_i$ took to exit the main loop in BYZCONSENSUS-EARLY. Let $r_m$ be the minimum round among $r_i$  

\begin{l1}
WEIGHT-UPDATE guarantees that at least $r_m - 1$ faulty processes will be detected per invocation.
\begin{proof}
$r_m$ is the number of rounds taken by the correct process that terminates earliest. From Claim 5  $|\cap BAD_{r_m}| \geq r_m - 1$, which means that all correct process have at least $r_m - 1$ common faulty processes in their $BAD$ set. Therefore, at least $r_m - 1$ faulty processes will be detected.
\end{proof}
\end{l1}


\begin{l1}
A correct process can never be in \textbf{faulty}. 
\begin{proof}
From Claim 5 for any $p_i$ $BAD_i$ only contains fault process. So a correct process can never be in \textit{\textbf{faulty}}.
\end{proof}
\end{l1}

\section{Conclusion}
This paper discussed a weighted version of the Byzantine Agreement Problem and provided solutions for the problem based on gradecast idea in a synchronous distributed system. We prove that the weighted version of gradecast has the same key properties as unweighted gradecast. Two algorithms are given to solve the WBA problem: one with early stopping property and another terminates in fewer rounds. 
These algorithms have applications in many systems in which
there are two classes of processes: trusted and untrusted
processes. Instead of tolerating any $f$ faults in the BA problem,
these algorithms tolerate failure of processes with total weight
less than $f/N$. For example, an implementation can now
tolerate more than $f$ faults of untrusted processes; but, fewer
than $f$ faults of trusted processes depending on the weight
assignment. A fault-tolerant method with guarantees on the number of faults it can detect has also been presented to update the weights at all the correct processes. This algorithm is useful for many applications where the agreement is required multiple times. Eventually all faulty processes can be detected out and BYZCONSENSUS-EARLY can reach agreement in only two rounds and BYZCONSENSUS-ANCHOR can reach agreement in one round. 


\begin{thebibliography}{00}
\bibitem{b1} G. Eason, B. Noble, and I. N. Sneddon, ``On certain integrals of Lipschitz-Hankel type involving products of Bessel functions,'' Phil. Trans. Roy. Soc. London, vol. A247, pp. 529--551, April 1955.
\bibitem{b2} J. Clerk Maxwell, A Treatise on Electricity and Magnetism, 3rd ed., vol. 2. Oxford: Clarendon, 1892, pp.68--73.
\bibitem{b3} I. S. Jacobs and C. P. Bean, ``Fine particles, thin films and exchange anisotropy,'' in Magnetism, vol. III, G. T. Rado and H. Suhl, Eds. New York: Academic, 1963, pp. 271--350.
\bibitem{b4} K. Elissa, ``Title of paper if known,'' unpublished.
\bibitem{b5} R. Nicole, ``Title of paper with only first word capitalized,'' J. Name Stand. Abbrev., in press.
\bibitem{b7} Y. Yorozu, M. Hirano, K. Oka, and Y. Tagawa, ``Electron spectroscopy studies on magneto-optical media and plastic substrate interface,'' IEEE Transl. J. Magn. Japan, vol. 2, pp. 740--741, August 1987 [Digests 9th Annual Conf. Magnetics Japan, p. 301, 1982].
\bibitem{b7} M. Young, The Technical Writer's Handbook. Mill Valley, CA: University Science, 1989.
\end{thebibliography}

[6] Garg's weighted byzantine paper 
[7]Paul Feldman and Silvio Micali. Optimal algorithms for byzantine agreement. In STOC ’88:
Proceedings of the twentieth annual ACM symposium on Theory of computing, pages 148–161,
New York, NY, USA, 1988. ACM.


[8]gradecast paper


\end{document}
